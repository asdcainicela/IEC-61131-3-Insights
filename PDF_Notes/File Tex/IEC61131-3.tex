\documentclass[20pt]{extarticle} % Usa extarticle para tamaños de fuente
\usepackage[utf8]{inputenc}
\usepackage[T1]{fontenc}
\usepackage{lmodern} % Fuente moderna que mejora los caracteres especiales
\usepackage{setspace} % Configuración para interlineado

% Paquetes para matemáticas
\usepackage{amsmath} % Entornos matemáticos avanzados
\usepackage{amssymb} % Símbolos matemáticos adicionales
\usepackage{amsfonts} % Fuentes matemáticas adicionales

% Paquetes para colores, gráficos y diseño de página
\usepackage{xcolor}  % Para colores
\usepackage{tikz}    % Para gráficos y degradados
\usepackage{pagecolor} % Para el color de fondo de toda la página
\usepackage{geometry} % Para ajustar las dimensiones de la página y márgenes
\usepackage{hyperref} % Para enlaces clicables
\hypersetup{
	colorlinks=true,
	linkcolor=blue,
	filecolor=magenta,      
	urlcolor=blue,
	pdftitle={IEC-61131-3-Insights},
}

\usepackage[plainfootsepline]{scrlayer-scrpage} % Para encabezados y pies de página personalizados

% Cambiar la fuente a Palatino (elegante y profesional)
\usepackage{mathpazo} % Usa Palatino como fuente principal
\usepackage{microtype} % Mejora la apariencia del texto con pdfLaTeX

% Configurar el tamaño de la hoja
\geometry{
	paperwidth=32cm, %25.4cm,
	paperheight=32cm, %31.75cm,
	margin=3cm, % Ajusta los márgenes según lo necesites
	top=4cm % Aumentamos el margen superior para separar el encabezado del cuerpo
}

% Definir los colores personalizados
\definecolor{ColorBg}{HTML}{F2F6FC}
\definecolor{ColorHeadLeft}{HTML}{F8F3ED}
\definecolor{ColorLineFoot}{HTML}{ded6d6}
\definecolor{ColorBlueText}{HTML}{1C3F60}
\definecolor{customcolor}{HTML}{1C3F60}

\definecolor{ColorBoxTitle}{HTML}{f2fef5}
\definecolor{ColorBox}{HTML}{f7fffe}
% Aplica el color de fondo de la página
\pagecolor{ColorBg}

% Define una capa con un degradado lineal en la cabecera de 25.4 cm x 3.17 cm
\DeclareNewLayer[
background,
topmargin,
addheight=3.17cm, % Altura fija para el encabezado
contents={
	\begin{tikzpicture}[remember picture, overlay]
		% Dibuja un rectángulo con el degradado lineal de izquierda a derecha
		\path[shade, left color=ColorHeadLeft, right color=ColorBg]
		(current page.north west) rectangle ([yshift=-3.17cm]current page.north east);
	\end{tikzpicture}
}
]{head.bg}

\AddLayersAtBeginOfPageStyle{scrheadings}{head.bg}

% Configuración del encabezado con TikZ para centrar verticalmente
\clearpairofpagestyles % Limpia encabezado y pie de página
\ihead{
	\begin{tikzpicture}[baseline]
		\node[anchor=center] at (0,0) {\includegraphics[height=1.05cm]{img/Beckhoff_Logo.png}};
	\end{tikzpicture}
} % Encabezado izquierdo con imagen
\chead{
	\begin{tikzpicture}[baseline]
		\node[anchor=center] at (0,0) {\fontsize{36pt}{36pt}\selectfont\textcolor{ColorBlueText}{\bf IEC 61131-3}};
	\end{tikzpicture}
} % Encabezado central con texto
\ohead{
	\begin{tikzpicture}[baseline]
		\node[anchor=center] at (0,0) {\includegraphics[height=2.2cm]{img/Codesys_Logo.png}};
	\end{tikzpicture}
} % Encabezado derecho con imagen

\DeclareNewLayer[
background,
bottommargin,
addheight=3cm, % Altura fija para el pie
contents={
	\begin{tikzpicture}[remember picture, overlay]
		% Dibuja un rectángulo con el degradado lineal de izquierda a derecha
		%\path[shade, left color=ColorLineFoot, right color=ColorLineFoot]
		(%current page.south west) rectangle ([yshift=2.5cm]current page.south east);
		\path[fill= ColorLineFoot]
		([yshift=2.55cm]current page.south west) rectangle ([yshift=2.5cm]current page.south east);
	\end{tikzpicture}
}
]{foot.bg}

% Activar las capas definidas
\AddLayersAtBeginOfPageStyle{scrheadings}{foot.bg}
% Pie de página personalizado
\cfoot{\thepage} % Centro vacío
\ofoot{
	\begin{tikzpicture}[baseline]
		\node[anchor=center] at (0,0) {\href{https://github.com/asdcainicela/IEC-61131-3-Insights}{\includegraphics[height=1.5cm]{img/GitHub_Logo.png}}};
	\end{tikzpicture}
} % Número de página a la derecha
\ifoot{\href{https://pe.linkedin.com/in/gerald-cainicela}{Gerald Cainicela}} % Texto y enlace a la izquierda

\usepackage{titlesec} % Para modificar el estilo de las secciones

% Configuración para las secciones
\titleformat{\section} % Formato de \section
{\centering\color{customcolor}\fontsize{40}{48}\bfseries} % Color, tamaño, centrado
{} % Etiqueta (vacío si no quieres incluir el número)
{0pt} % Separación entre la etiqueta y el título
{}

% Configuración para las subsecciones
\titleformat{\subsection} % Formato de \subsection
{\color{customcolor}\fontsize{31}{37}\bfseries} % Color, tamaño, alineación (no centrado)
{} % Etiqueta (vacío si no quieres incluir el número)
{0pt} % Separación entre la etiqueta y el título
{}

% Definir comandos para resaltar palabras en rojo
\newcommand{\imp}[1]{\textcolor{red}{#1}}

\usepackage{listings, multicol}
  
\setlength{\parskip}{15pt} % Espaciado entre párrafos
\setlength{\parindent}{0pt} % Sin indentación al inicio de los párrafos
%https://github.com/Ydalirsson/latex-listings-ST/blob/main/codestyle.tex

%New colors defined below
\definecolor{codegreen}{rgb}{0,0.6,0}
\definecolor{codegray}{rgb}{0.5,0.5,0.5}
\definecolor{codepurple}{rgb}{0.502,0.502,0.0}
\definecolor{backcolour}{rgb}{0.95,0.95,0.95}
\definecolor{ColorFrame}{HTML}{A8A3F0}

\lstdefinelanguage{ST}
{
	morekeywords={
		case,of,if,then,end_if,end_case,super,function_block,extends,var,
		constant, byte,,end_var,var_input, real,bool,var_output,
		dint,udint,word,dword,array, of,uint,not,adr, program, for, end_for, while, do, end_while, repeat, end_repeat, until, to, by, else, elsif, var_in_out, and_then, or_else, method, and, or, xor, pointer, int, limit, <,>,=, min, max, mux, sel
	},
	otherkeywords={
		:, :=, <>,;,\,.,\[,\],\^,1,2,3,4,5,6,7,8,9,0,TRUE, FALSE, \#, \{attribute,  \'hide\'\}
	},
	keywords=[1]{
		case,of,if,then,end_if,end_case,super,function_block,extends,var,
		constant, byte,,end_var,var_input, real,bool,var_output,
		dint,udint,word,dword,array, of,uint,not,adr, :, :=, <>,;,\,.,\[,\],\^,program, for, end_for, while, do, end_while, repeat, end_repeat, until, to, by, else, elsif, var_in_out, and_then, or_else, method, and, or, xor, pointer, int, limit, <, >, =, min, max, mux, sel
	},
	keywordstyle=[1]\color{blue},
	keywords=[2]{
		1,2,3,4,5,6,7,8,9,0, TRUE, FALSE, \#
	},
	keywordstyle=[2]\color{codepurple},
	keywords=[3]{
		\{attribute,  \'hide\'\}
	},
	keywordstyle=[3]\color{codegray},
	sensitive=false,
	morecomment=[l]{//}, 
	morecomment=[s]{(*}{*)},
	morestring=[b]{"},
	morestring=[b]{'}
}

\lstset{
	language={ST},
	backgroundcolor=\color{white}, % Fondo blanco
	commentstyle=\color{codegreen}, %\textit ,
	keywordstyle=\color{blue},
	numberstyle=\tiny\color{codegray},
	stringstyle=\color{codepurple},
	basicstyle= \normalsize ,%\small,% \footnotesize  , % Tamaño de fuente
	breakatwhitespace=true,         
	breaklines=true,                 
	%captionpos=b,                    
	keepspaces=false,                 
	numbersep=5pt,                   
	showspaces=false,                
	showstringspaces=false,
	showtabs=false,                   
	tabsize=2,
	aboveskip=15pt,  
	belowskip=0pt,  % Sin espacio después
	xleftmargin=0pt,  % Sin margen a la izquierda
	xrightmargin=0pt,  % Sin margen a la derecha
	frame=tb,   % Marco alrededor del código 
	rulecolor=\color{ColorFrame},  % Color de las líneas del marco
	columns=flexible, % Espaciado flexible entre columnas
	escapeinside={(*@}{@*)}, % Allows embedding LaTeX commands
	literate=
	*{0}{{{\color{codepurple}0}}}{1}%
	{1}{{{\color{codepurple}1}}}{1}%
	{2}{{{\color{codepurple}2}}}{1}%
	{3}{{{\color{codepurple}3}}}{1}%
	{4}{{{\color{codepurple}4}}}{1}%
	{5}{{{\color{codepurple}5}}}{1}%
	{6}{{{\color{codepurple}6}}}{1}%
	{7}{{{\color{codepurple}7}}}{1}%
	{8}{{{\color{codepurple}8}}}{1}%
	{9}{{{\color{codepurple}9}}}{1}
}



\setlength{\parindent}{0pt}  % Elimina la indentación de párrafos

\usepackage{wrapfig} % Paquete necesario
\usepackage{lipsum}
\usepackage[x11names,table]{xcolor}  % Asegúrate de usar 'table' 
 
 \usepackage{colortbl} % Para los colores en las tablas
%% Automtizaciones
\newcommand{\myWrapTable}[2]{
	\begin{wraptable}{r}{0.3\textwidth} % "l" para alinear a la izquierda, "0.4\textwidth" es el ancho de la tabla
		\centering
		\resizebox{0.2\textwidth}{!}{%
			\begin{tabular}{#1} % Aquí se pasa la estructura de la tabla como argumento
				#2 % Aquí se pasa el contenido de las filas como argumento
			\end{tabular}
		}
		\vspace{-25pt} % Ajusta el espacio inferior
	\end{wraptable}
}

% Definimos un nuevo comando que usa lstinline y fcolorbox
\newcommand{\highlight}[1]{\fcolorbox{white}{white}{\lstinline|#1|}}
\newcommand{\Highlight}[1]{\lstinline|#1|}


\usepackage[most]{tcolorbox}
\newtcolorbox{Box1}[2][]{
	lower separated=false,
	colback =ColorBox, % Fondo de contenido claro
	colframe=white, fonttitle=\bfseries,
	colbacktitle=ColorBoxTitle,
	coltitle=black,
	enhanced,
	attach boxed title to top left={xshift=0.5cm,yshift=-2mm},
	title=#2,#1
}
\definecolor{ColorDebugFill}{HTML}{fff3e7}
\definecolor{ColorDebugFrame}{HTML}{ab8975}
\newcommand{\debugval}[1]{\fcolorbox{ColorDebugFrame}{ColorDebugFill}{\small{#1}}\ } % Command for orange debug values
\begin{document}
	
	\arrayrulecolor[HTML]{3798a2}
	\setlength{\tabcolsep}{2\tabcolsep} % Duplica el espacio entre columnas
	\setlength\intextsep{-0.1ex}
	%\tableofcontents
	%\newpage
	\setstretch{1.5}
	\section{ADDRESS OPERATORS}
\subsection{ADR}

\begin{Box1}{Syntax} 
VAR
<address name> : DWORD | LWORD | POINTER TO < basis data type> | \_\_XWORD ;
END\_VAR

<address name> := ADR( <variable name> );

\end{Box1}	

\subsection{Content operator}
\subsection{BITADR}
	\section{SELECTION OPERATORS}


\subsection{MAX}

The IEC operator is used for the maximum function. It yields the greatest value of all inputs.

\begin{Box1}{Syntax}
	OUT := MAX(IN0, IN1, <further inputs>)
\end{Box1}	

\textbf{Permitted data types:} All


\begin{lstlisting}[language=ST ]
	PROGRAM Selection_Operators
	VAR
		rRoomTemp : REAL := 22.5;   // Room temperature
		rExtTemp : REAL := 18.3;    // External temperature
		rMaxTemp : REAL;    // Maximum temperature
		iMaxAge : INT;  // Maximum age of the plant workers
	END_VAR
	
	// Get the highest temperature between the two
	rMaxTemp(*@\debugval{22.5}@*) := MAX(rRoomTemp(*@\debugval{22.5}@*), rExtTemp(*@\debugval{18.3}@*));
	
	// Result is the maximum age of the plant workers
	iMaxAge(*@\debugval{39}@*) := MAX(32, 25, 36, 24, 18, 39, 30);  
\end{lstlisting}




\subsection{LIMIT}
The IEC selection operator is used for limitation.

\begin{Box1}{Syntax}
	\Highlight{OUT :=  LIMIT}(Min, IN, Max) \hspace{5pt}
\end{Box1}	
%	means: \Highlight{OUT := MIN(MAX (IN,} Min), Max) 
\textbf{Permitted data types:} All.

%\resizebox{0.9\textwidth}{!}{%
\begin{tabular}{lll} % Aquí se pasa la estructura de la tabla como argumento
	\hline
	\textbf{Input Value (IN)} & \textbf{Output (OUT)} & \textbf{Explanation} \\
	\hline
	IN $<$ Min & OUT = Min & The value is below the lower limit. \\ 
	\hline
	Min $\leq$ IN $\leq$ Max & OUT = IN & The value is within the specified range. \\ 
	\hline
	IN $>$ Max & OUT = Max & The value exceeds the upper limit. \\ 
	\hline
\end{tabular}
%}
\newpage
The \highlight{LIMIT} function ensures that the output remains within a defined range. The following table demonstrates how it behaves with different input values.

\begin{lstlisting}[language=ST ]
	PROGRAM Selection_Operators
		iInPressure : INT;    // Raw pressure sensor value (input)
		iPressure : INT;          // Adjusted pressure value (output)
	END_VAR
	VAR CONSTANT 
		cMinPressure : INT := 100;   // Minimum allowed pressure
		cMaxPressure : INT := 150;   // Maximum allowed pressure
	END_VAR
	
	iPressure(*@\debugval{100}@*) := LIMIT(cMinPressure(*@\debugval{100}@*), iInPressure(*@\debugval{80}@*), cMaxPressure(*@\debugval{150}@*));
\end{lstlisting} 

%\resizebox{1\textwidth}{!}



The LIMIT operator, conceptually is: 
\fcolorbox{white}{white}{\Highlight{OUT := MIN(MAX}(IN, Min), Max)}.


\subsection{MIN}

The IEC operator is used for the minimum function. It yields the least value of all inputs. 

\begin{Box1}{Syntax}
	OUT := MIN(IN0, IN1, <further inputs>)
\end{Box1}	

\textbf{Permitted data types:} All.

\begin{lstlisting}[language=ST ]
	PROGRAM Selection_Operators
	VAR
	rRoomTemp : REAL := 22.5;   // Room temperature 
	rExtTemp : REAL := 18.3;    // External temperature
	rMinTemp : REAL;    // Minimum temperature
	iMinAge : INT;  // Minimum age of the plant workers
	END_VAR
	
	// Get the lowest temperature between the two
	rMinTemp(*@\debugval{18.3}@*) := MIN(rRoomTemp(*@\debugval{22.5}@*), rExtTemp(*@\debugval{18.3}@*));
	
	// Result is the minimum age of the plant workers
	iMinAge(*@\debugval{18}@*) := MIN(32, 25, 36, 24, 18, 39, 30);  
	// Result is 18, the minimum age of the plant workers
\end{lstlisting}


\subsection{MUX}

The IEC operator is used as a multiplexer.

\begin{Box1}{Syntax}
	OUT := MUX(K, IN0, ..., INn)
\end{Box1}

This means OUT takes the value of IN\_K, where K is the specified index.

\textbf{Permitted data types:}
\begin{itemize}
\item K:  e  BYTE, WORD, DWORD, LWORD, SINT, USINT, INT, UINT, DINT, LINT, ULINT, UDINT.

\item  IN0, ..., INn and OUT: Any identical data type.% Make sure that variables with the same type are used at all three positions, especially when using user-defined data types. The compiler checks the type equality and issues compilation errors. In particular, the allocation of instances of a function block to interface (variables) is not supported.
	
%The MUX operator selects the K-th value from a set of values, where the first value corresponds to K=0. If K exceeds the number of additional inputs (n), both TwinCAT and CODESYS pass the last value (INn).


%\textbf{Important!}\\
%Both CODESYS and TwinCAT optimize runtime by calculating only the expression preceding IN\_K. However, CODESYS computes all branches in simulation mode, while TwinCAT maintains the optimization consistently.

\end{itemize}
\newpage

\begin{lstlisting}[language=ST]
	PROGRAM Selection_Operators
	VAR
		nSelectedValue : INT;   // The selected value
		iK : INT := 2;          // The index (K), initialized with 2
	END_VAR
	
	// iK=2, so the selected value will be 30
	nSelectedValue(*@\debugval{30}@*) := MUX(iK(*@\debugval{2}@*), 10, 20, 30, 40, 50);  
\end{lstlisting}
\begin{wraptable}{r}{0.6 \textwidth} % "l" para alinear a la izquierda, "0.4\textwidth" es el ancho de la tabla
	\centering
	\resizebox{0.55\textwidth}{!}{%
		 \begin{tabular}{ccc}
		 	\hline
		 	\textbf{(Index)} & \textbf{Values } & \textbf{(OUT)} \\
		 	\hline
		 	\highlight{iK}  & IN0, IN1, IN2, IN3, IN4 & \highlight{nSelectedValue} \\
		 	\hline
		 	\highlight{-iK} & 10, 20, 30, 40, 50 & 50 \\
		 	\hline
		 	0 & 10, 20, 30, 40, 50 & 10 \\
		 	\hline
		 	1 & 10, 20, 30, 40, 50 & 20 \\
		 	\hline
		 	2 & 10, 20, 30, 40, 50 & 30 \\
		 	\hline
		 	3 & 10, 20, 30, 40, 50 & 40 \\
		 	\hline
		 	4 & 10, 20, 30, 40, 50 & 50 \\
		 	\hline
		 	\highlight{iK > 4} & 10, 20, 30, 40, 50 & 50 \\
		 	\hline
		 \end{tabular}
		}
	\vspace{-25pt} % Ajusta el espacio inferior
\end{wraptable}


When the index \highlight{iK} is negative (e.g.,  \highlight{iK = -1} ), the MUX operator selects the last value (IN4).

When \highlight{iK} exceeds the number of available inputs (e.g.,  \highlight{iK} = 5 for a 5-input MUX), the operator will also select the last value (IN4).

\subsection{SEL}

The IEC operator is used for bitwise selection.

\begin{Box1}{Syntax}
	OUT := SEL(G, IN0, IN1)
\end{Box1}


\begin{wraptable}{r}{0.25 \textwidth} % "l" para alinear a la izquierda, "0.4\textwidth" es el ancho de la tabla
	\centering
	\resizebox{0.2\textwidth}{!}{%
		\begin{tabular}{cc}
			\hline
			\textbf{G} &  \textbf{OUT} \\
			\hline
			\hline
			\highlight{FALSE} & IN0\\
			\hline
			\highlight{TRUE} & IN1\\
			\hline
		\end{tabular}
	}
	\vspace{-25pt} % Ajusta el espacio inferior
\end{wraptable}

\textbf{Permitted data types:}
\begin{itemize}
	\item IN0, ..., INn and OUT: any identical data type. %Make sure that variables of the identical type are used at all three positions, especially when using user-defined data types. The compiler checks for type identity and returns any compile errors. The assignment of function block instances to interface variables is specifically not supported.
	\item G: BOOL
\end{itemize}

%\textbf{Important:}
%TwinCAT and CODESYS do not compute an expression that precedes IN0 when G is TRUE, nor do they compute an expression that precedes IN1 when G is FALSE, and in graphical programming languages, the expressions at IN0 and IN1 are computed independently of the G input when a function block, a jump, a return, a branch, or edge detection is connected upstream.


\begin{lstlisting}[language=ST]
	VAR
		iVarSel : INT; // Result of SEL
	END_VAR
	
	iVarSel := SEL(FALSE, 3, 4); (* Result: 3 *)
	iVarSel := SEL(TRUE, 3, 4); (* Result: 4 *)
\end{lstlisting}



	%\section{BITSTRING OPERATORS}

\subsection{AND} 

\myWrapTable{ccc}{ 
	\hline
	A & B & A \imp{AND} B \\ 
	\hline
	0 & 0 & 0 \\ 
	\hline
	0 & 1 & 0 \\ 
	\hline
	1 & 0 & 0 \\ 
	\hline
	1 & 1 & 1 \\ 
	\hline
}

The operator is used for bitwise \imp{AND} operation of bit operands.

Permitted data types: BOOL, BYTE, WORD, DWORD, LWORD.

\begin{lstlisting}[language=ST ]
	PROGRAM Bitstring_Operators
	VAR 
		wVarAnd: WORD;
	END_VAR
	
	wVarAnd := 2#1001_0011 AND 2#1000_1010;
	(* Result in wVarAnd: 2#1000_0010 *)
\end{lstlisting}	  

\subsection{AND\_THEN}

The operator is an extension of the IEC 61131-3 standard, used
exclusively in \imp{Structured Text (ST)} for \imp{AND} operation with \imp{shortcircuit evaluation}
on \imp{BOOL} and \imp{BIT} operands.

When all operands yield \imp{TRUE}, the result of the operands also yield \imp{TRUE}; otherwise \imp{FALSE}.


\begin{lstlisting}[language=ST ]
	PROGRAM Bitstring_Operators
	VAR 
		pxSensor : POINTER TO BOOL;       // Pointer to a sensor 
		xAlarm : BOOL;                   // Variable to activate the alarm
	END_VAR
	
	(* pxSensor := ADR(xSomethingValue);  Is NULL if not initialized *)
	IF (pxSensor <> 0 AND_THEN pxSensor^) THEN 
		(* Additional logic can be implemented below *)
		xAlarm := TRUE;
	END_IF
\end{lstlisting} 

%\begin{enumerate}
	%\item 
	Check if the pointer \highlight{pxSensor} is not null. If it is null, the rest of the condition is not evaluated.
	%\item
	 The \highlight{pxSensor^} is only evaluated if \highlight{pxSensor <> 0} is \highlight{TRUE}, avoiding dereference errors.
%\end{enumerate}

Using \highlight{AND\_THEN} prevents the program from accessing \highlight{pxSensor^} when \highlight{pxSensor} is 0, avoiding runtime errors, making it safer and more efficient than using \highlight{AND}.
\subsection{OR\_ELSE} 
The operator is an extension of the IEC 61131-3 standard, used
exclusively in \imp{Structured Text (ST)} for \imp{OR} operation with \imp{shortcircuit evaluation}
on \imp{BOOL} and \imp{BIT} operands.

When at least one of the operands yields \imp{TRUE}, the result of the operation also yields \imp{TRUE}; otherwise \imp{FALSE}.

\begin{lstlisting}[language=ST ]
	FUNCTION_BLOCK FB_OrElse
	VAR
		iCounter : INT;
	END_VAR
\end{lstlisting}
 
\newpage
\begin{lstlisting}[language=ST ]	
	(* Method of the Function Block FB_OrElse *)
	METHOD TestMethod : BOOL
	iCounter := iCounter + 1;
	TestMethod := TRUE; (* Set the method's return value to TRUE *)
\end{lstlisting} 

\begin{lstlisting}[language=ST ]
	PROGRAM Bitstring_Operators
	VAR 
		fbSampleOrElse : FB_OrElse; //Instance of the FB_OrElse function block 
		xResult : BOOL;
		xVar : BOOL;
	END_VAR
	
	xResult := xVar OR_ELSE fbSampleOrElse.TestMethod();
\end{lstlisting} 

\begin{enumerate}
	\item If \highlight{xVar} is \highlight{TRUE}, the method does not execute, and the counter iCounter does not increment.
	\item If \highlight{xVar} is \highlight{FALSE}, the method executes, and the counter iCounter increments by 1.
\end{enumerate}

With \highlight{OR_ELSE}, if any operand is \highlight{TRUE}, the rest of the expressions are not evaluated, unlike the regular \highlight{OR} operator.

\subsection{OR} 

\myWrapTable{ccc}{ 
	\hline
	A & B & A \imp{OR} B \\ 
	\hline
	0 & 0 & 0 \\ 
	\hline
	0 & 1 & 1 \\ 
	\hline
	1 & 0 & 1 \\ 
	\hline
	1 & 1 & 1 \\ 
	\hline
}

The IEC operator is used for bitwise \imp{OR} operation of bit operands.

Permitted data types: BOOL, BYTE, WORD, DWORD, LWORD.

\vspace{15pt}

\begin{lstlisting}[language=ST ]
	PROGRAM Bitstring_Operators
	VAR 
		wVarOr: WORD;
	END_VAR
	
	wVarOr:= 2#1001_0011 OR 2#1000_1010;
	(* Result in wVarOr: 2#1001_1011 *)
\end{lstlisting}	
 

\subsection{XOR} 

	\myWrapTable{ccc}{ 
		\hline
		A & B & A \imp{XOR} B \\ 
		\hline
		0 & 0 & 0 \\ 
		\hline
		0 & 1 & 1 \\ 
		\hline
		1 & 0 & 1 \\ 
		\hline
		1 & 1 & 0 \\ 
		\hline
	}
	The IEC operator is used for bitwise \imp{XOR} operation of bit operands.
	
	Permitted data types: BOOL, BYTE, WORD, DWORD, LWORD. 
	
	Note the following behavior of the XOR POU in extended form (more than two inputs): compares the inputs in pairs and then the corresponding results (according to the standard, but not necessarily according to expectations).
	
	\begin{lstlisting}[language=ST ]
		PROGRAM Bitstring_Operators
		VAR 
			wVarXor: WORD;
		END_VAR
		
		wVarXor := 2#1001_0011 XOR 2#1000_1010;
		(* Result in wVarXor: 2#0001_1001 *)
	\end{lstlisting}



\subsection{NOT} 

	\myWrapTable{cc}{ 
		\hline
		A &  \imp{NOT} A \\ 
		\hline
		 0 & 1 \\ 
		\hline
		 1 & 0 \\ 
		\hline
	}
	
	
	The IEC operator is used for bitwise \imp{NOT} of bit operands.
	
	Permitted data types: BOOL, BYTE, WORD, DWORD, LWORD.
	
	\begin{lstlisting}[language=ST ]
		PROGRAM Bitstring_Operators
		VAR 
			wVarNot: WORD;
		END_VAR
		
		wVarNot := NOT 2#1001_0011;
		(* Result in wVarNot: 2#0110_1100 *)
	\end{lstlisting}	 


	
\end{document}