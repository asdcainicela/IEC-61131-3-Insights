\section{BITSTRING OPERATORS}

\subsection{AND} 

\myWrapTable{ccc}{ 
	\hline
	A & B & A \imp{AND} B \\ 
	\hline
	0 & 0 & 0 \\ 
	\hline
	0 & 1 & 0 \\ 
	\hline
	1 & 0 & 0 \\ 
	\hline
	1 & 1 & 1 \\ 
	\hline
}

The operator is used for bitwise \imp{AND} operation of bit operands.

Permitted data types: BOOL, BYTE, WORD, DWORD, LWORD.

\begin{lstlisting}[language=ST ]
	PROGRAM Bitstring_Operators
	VAR 
		wVarAnd: WORD;
	END_VAR
	
	wVarAnd := 2#1001_0011 AND 2#1000_1010;
	(* Result in wVarAnd: 2#1000_0010 *)
\end{lstlisting}	  

\subsection{AND\_THEN}

The operator is an extension of the IEC 61131-3 standard, used
exclusively in \imp{Structured Text (ST)} for \imp{AND} operation with \imp{shortcircuit evaluation}
on \imp{BOOL} and \imp{BIT} operands.

When all operands yield \imp{TRUE}, the result of the operands also yield \imp{TRUE}; otherwise \imp{FALSE}.


\begin{lstlisting}[language=ST ]
	PROGRAM Bitstring_Operators
	VAR 
		pxSensor : POINTER TO BOOL;       // Pointer to a sensor 
		xAlarm : BOOL;                   // Variable to activate the alarm
	END_VAR
	
	(* pxSensor := ADR(xSomethingValue);  Is NULL if not initialized *)
	IF (pxSensor <> 0 AND_THEN pxSensor^) THEN 
		(* Additional logic can be implemented below *)
		xAlarm := TRUE;
	END_IF
\end{lstlisting} 

%\begin{enumerate}
	%\item 
	Check if the pointer \highlight{pxSensor} is not null. If it is null, the rest of the condition is not evaluated.
	%\item
	 The \highlight{pxSensor^} is only evaluated if \highlight{pxSensor <> 0} is \highlight{TRUE}, avoiding dereference errors.
%\end{enumerate}

Using \highlight{AND\_THEN} prevents the program from accessing \highlight{pxSensor^} when \highlight{pxSensor} is 0, avoiding runtime errors, making it safer and more efficient than using \highlight{AND}.
\subsection{OR\_ELSE} 
The operator is an extension of the IEC 61131-3 standard, used
exclusively in \imp{Structured Text (ST)} for \imp{OR} operation with \imp{shortcircuit evaluation}
on \imp{BOOL} and \imp{BIT} operands.

When at least one of the operands yields \imp{TRUE}, the result of the operation also yields \imp{TRUE}; otherwise \imp{FALSE}.

\begin{lstlisting}[language=ST ]
	FUNCTION_BLOCK FB_OrElse
	VAR
		iCounter : INT;
	END_VAR
\end{lstlisting}
 
\newpage
\begin{lstlisting}[language=ST ]	
	(* Method of the Function Block FB_OrElse *)
	METHOD TestMethod : BOOL
	iCounter := iCounter + 1;
	TestMethod := TRUE; (* Set the method's return value to TRUE *)
\end{lstlisting} 

\begin{lstlisting}[language=ST ]
	PROGRAM Bitstring_Operators
	VAR 
		fbSampleOrElse : FB_OrElse; //Instance of the FB_OrElse function block 
		xResult : BOOL;
		xVar : BOOL;
	END_VAR
	
	xResult := xVar OR_ELSE fbSampleOrElse.TestMethod();
\end{lstlisting} 

\begin{enumerate}
	\item If \highlight{xVar} is \highlight{TRUE}, the method does not execute, and the counter iCounter does not increment.
	\item If \highlight{xVar} is \highlight{FALSE}, the method executes, and the counter iCounter increments by 1.
\end{enumerate}

With \highlight{OR_ELSE}, if any operand is \highlight{TRUE}, the rest of the expressions are not evaluated, unlike the regular \highlight{OR} operator.

\subsection{OR} 

\myWrapTable{ccc}{ 
	\hline
	A & B & A \imp{OR} B \\ 
	\hline
	0 & 0 & 0 \\ 
	\hline
	0 & 1 & 1 \\ 
	\hline
	1 & 0 & 1 \\ 
	\hline
	1 & 1 & 1 \\ 
	\hline
}

The IEC operator is used for bitwise \imp{OR} operation of bit operands.

Permitted data types: BOOL, BYTE, WORD, DWORD, LWORD.

\vspace{15pt}

\begin{lstlisting}[language=ST ]
	PROGRAM Bitstring_Operators
	VAR 
		wVarOr: WORD;
	END_VAR
	
	wVarOr:= 2#1001_0011 OR 2#1000_1010;
	(* Result in wVarOr: 2#1001_1011 *)
\end{lstlisting}	
 

\subsection{XOR} 

	\myWrapTable{ccc}{ 
		\hline
		A & B & A \imp{XOR} B \\ 
		\hline
		0 & 0 & 0 \\ 
		\hline
		0 & 1 & 1 \\ 
		\hline
		1 & 0 & 1 \\ 
		\hline
		1 & 1 & 0 \\ 
		\hline
	}
	The IEC operator is used for bitwise \imp{XOR} operation of bit operands.
	
	Permitted data types: BOOL, BYTE, WORD, DWORD, LWORD. 
	
	Note the following behavior of the XOR POU in extended form (more than two inputs): compares the inputs in pairs and then the corresponding results (according to the standard, but not necessarily according to expectations).
	
	\begin{lstlisting}[language=ST ]
		PROGRAM Bitstring_Operators
		VAR 
			wVarXor: WORD;
		END_VAR
		
		wVarXor := 2#1001_0011 XOR 2#1000_1010;
		(* Result in wVarXor: 2#0001_1001 *)
	\end{lstlisting}



\subsection{NOT} 

	\myWrapTable{cc}{ 
		\hline
		A &  \imp{NOT} A \\ 
		\hline
		 0 & 1 \\ 
		\hline
		 1 & 0 \\ 
		\hline
	}
	
	
	The IEC operator is used for bitwise \imp{NOT} of bit operands.
	
	Permitted data types: BOOL, BYTE, WORD, DWORD, LWORD.
	
	\begin{lstlisting}[language=ST ]
		PROGRAM Bitstring_Operators
		VAR 
			wVarNot: WORD;
		END_VAR
		
		wVarNot := NOT 2#1001_0011;
		(* Result in wVarNot: 2#0110_1100 *)
	\end{lstlisting}	 

