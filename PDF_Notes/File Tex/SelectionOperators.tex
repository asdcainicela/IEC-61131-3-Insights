\section{SELECTION OPERATORS}


\subsection{MAX}

The IEC operator is used for the maximum function. It yields the greatest value of all inputs.

\begin{Box1}{Syntax}
	OUT := MAX(IN0, IN1, <further inputs>)
\end{Box1}	

\textbf{Permitted data types:} All


\begin{lstlisting}[language=ST ]
	PROGRAM Selection_Operators
	VAR
		rRoomTemp : REAL := 22.5;   // Room temperature
		rExtTemp : REAL := 18.3;    // External temperature
		rMaxTemp : REAL;    // Maximum temperature
		iMaxAge : INT;  // Maximum age of the plant workers
	END_VAR
	
	// Get the highest temperature between the two
	rMaxTemp(*@\debugval{22.5}@*) := MAX(rRoomTemp(*@\debugval{22.5}@*), rExtTemp(*@\debugval{18.3}@*));
	
	// Result is the maximum age of the plant workers
	iMaxAge(*@\debugval{39}@*) := MAX(32, 25, 36, 24, 18, 39, 30);  
\end{lstlisting}




\subsection{LIMIT}
The IEC selection operator is used for limitation.

\begin{Box1}{Syntax}
	\Highlight{OUT :=  LIMIT}(Min, IN, Max) \hspace{5pt}
\end{Box1}	
%	means: \Highlight{OUT := MIN(MAX (IN,} Min), Max) 
\textbf{Permitted data types:} All.

%\resizebox{0.9\textwidth}{!}{%
\begin{tabular}{lll} % Aquí se pasa la estructura de la tabla como argumento
	\hline
	\textbf{Input Value (IN)} & \textbf{Output (OUT)} & \textbf{Explanation} \\
	\hline
	IN $<$ Min & OUT = Min & The value is below the lower limit. \\ 
	\hline
	Min $\leq$ IN $\leq$ Max & OUT = IN & The value is within the specified range. \\ 
	\hline
	IN $>$ Max & OUT = Max & The value exceeds the upper limit. \\ 
	\hline
\end{tabular}
%}
\newpage
The \highlight{LIMIT} function ensures that the output remains within a defined range. The following table demonstrates how it behaves with different input values.

\begin{lstlisting}[language=ST ]
	PROGRAM Selection_Operators
		iInPressure : INT;    // Raw pressure sensor value (input)
		iPressure : INT;          // Adjusted pressure value (output)
	END_VAR
	VAR CONSTANT 
		cMinPressure : INT := 100;   // Minimum allowed pressure
		cMaxPressure : INT := 150;   // Maximum allowed pressure
	END_VAR
	
	iPressure(*@\debugval{100}@*) := LIMIT(cMinPressure(*@\debugval{100}@*), iInPressure(*@\debugval{80}@*), cMaxPressure(*@\debugval{150}@*));
\end{lstlisting} 

%\resizebox{1\textwidth}{!}



The LIMIT operator, conceptually is: 
\fcolorbox{white}{white}{\Highlight{OUT := MIN(MAX}(IN, Min), Max)}.


\subsection{MIN}

The IEC operator is used for the minimum function. It yields the least value of all inputs. 

\begin{Box1}{Syntax}
	OUT := MIN(IN0, IN1, <further inputs>)
\end{Box1}	

\textbf{Permitted data types:} All.

\begin{lstlisting}[language=ST ]
	PROGRAM Selection_Operators
	VAR
	rRoomTemp : REAL := 22.5;   // Room temperature 
	rExtTemp : REAL := 18.3;    // External temperature
	rMinTemp : REAL;    // Minimum temperature
	iMinAge : INT;  // Minimum age of the plant workers
	END_VAR
	
	// Get the lowest temperature between the two
	rMinTemp(*@\debugval{18.3}@*) := MIN(rRoomTemp(*@\debugval{22.5}@*), rExtTemp(*@\debugval{18.3}@*));
	
	// Result is the minimum age of the plant workers
	iMinAge(*@\debugval{18}@*) := MIN(32, 25, 36, 24, 18, 39, 30);  
	// Result is 18, the minimum age of the plant workers
\end{lstlisting}


\subsection{MUX}

The IEC operator is used as a multiplexer.

\begin{Box1}{Syntax}
	OUT := MUX(K, IN0, ..., INn)
\end{Box1}

This means OUT takes the value of IN\_K, where K is the specified index.

\textbf{Permitted data types:}
\begin{itemize}
\item K:  e  BYTE, WORD, DWORD, LWORD, SINT, USINT, INT, UINT, DINT, LINT, ULINT, UDINT.

\item  IN0, ..., INn and OUT: Any identical data type.% Make sure that variables with the same type are used at all three positions, especially when using user-defined data types. The compiler checks the type equality and issues compilation errors. In particular, the allocation of instances of a function block to interface (variables) is not supported.
	
%The MUX operator selects the K-th value from a set of values, where the first value corresponds to K=0. If K exceeds the number of additional inputs (n), both TwinCAT and CODESYS pass the last value (INn).


%\textbf{Important!}\\
%Both CODESYS and TwinCAT optimize runtime by calculating only the expression preceding IN\_K. However, CODESYS computes all branches in simulation mode, while TwinCAT maintains the optimization consistently.

\end{itemize}
\newpage

\begin{lstlisting}[language=ST]
	PROGRAM Selection_Operators
	VAR
		nSelectedValue : INT;   // The selected value
		iK : INT := 2;          // The index (K), initialized with 2
	END_VAR
	
	// iK=2, so the selected value will be 30
	nSelectedValue(*@\debugval{30}@*) := MUX(iK(*@\debugval{2}@*), 10, 20, 30, 40, 50);  
\end{lstlisting}
\begin{wraptable}{r}{0.6 \textwidth} % "l" para alinear a la izquierda, "0.4\textwidth" es el ancho de la tabla
	\centering
	\resizebox{0.55\textwidth}{!}{%
		 \begin{tabular}{ccc}
		 	\hline
		 	\textbf{(Index)} & \textbf{Values } & \textbf{(OUT)} \\
		 	\hline
		 	\highlight{iK}  & IN0, IN1, IN2, IN3, IN4 & \highlight{nSelectedValue} \\
		 	\hline
		 	\highlight{-iK} & 10, 20, 30, 40, 50 & 50 \\
		 	\hline
		 	0 & 10, 20, 30, 40, 50 & 10 \\
		 	\hline
		 	1 & 10, 20, 30, 40, 50 & 20 \\
		 	\hline
		 	2 & 10, 20, 30, 40, 50 & 30 \\
		 	\hline
		 	3 & 10, 20, 30, 40, 50 & 40 \\
		 	\hline
		 	4 & 10, 20, 30, 40, 50 & 50 \\
		 	\hline
		 	\highlight{iK > 4} & 10, 20, 30, 40, 50 & 50 \\
		 	\hline
		 \end{tabular}
		}
	\vspace{-25pt} % Ajusta el espacio inferior
\end{wraptable}


When the index \highlight{iK} is negative (e.g.,  \highlight{iK = -1} ), the MUX operator selects the last value (IN4).

When \highlight{iK} exceeds the number of available inputs (e.g.,  \highlight{iK} = 5 for a 5-input MUX), the operator will also select the last value (IN4).

\subsection{SEL}

The IEC operator is used for bitwise selection.

\begin{Box1}{Syntax}
	OUT := SEL(G, IN0, IN1)
\end{Box1}


\begin{wraptable}{r}{0.25 \textwidth} % "l" para alinear a la izquierda, "0.4\textwidth" es el ancho de la tabla
	\centering
	\resizebox{0.2\textwidth}{!}{%
		\begin{tabular}{cc}
			\hline
			\textbf{G} &  \textbf{OUT} \\
			\hline
			\hline
			\highlight{FALSE} & IN0\\
			\hline
			\highlight{TRUE} & IN1\\
			\hline
		\end{tabular}
	}
	\vspace{-25pt} % Ajusta el espacio inferior
\end{wraptable}

\textbf{Permitted data types:}
\begin{itemize}
	\item IN0, ..., INn and OUT: any identical data type. %Make sure that variables of the identical type are used at all three positions, especially when using user-defined data types. The compiler checks for type identity and returns any compile errors. The assignment of function block instances to interface variables is specifically not supported.
	\item G: BOOL
\end{itemize}

%\textbf{Important:}
%TwinCAT and CODESYS do not compute an expression that precedes IN0 when G is TRUE, nor do they compute an expression that precedes IN1 when G is FALSE, and in graphical programming languages, the expressions at IN0 and IN1 are computed independently of the G input when a function block, a jump, a return, a branch, or edge detection is connected upstream.


\begin{lstlisting}[language=ST]
	VAR
		iVarSel : INT; // Result of SEL
	END_VAR
	
	iVarSel := SEL(FALSE, 3, 4); (* Result: 3 *)
	iVarSel := SEL(TRUE, 3, 4); (* Result: 4 *)
\end{lstlisting}


